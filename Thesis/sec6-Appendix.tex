\onecolumn
\section{Appendix}
\subsection{Einstein Tensor Definition}
\label{apx:einsteintensor}
We first define the Riemann tensor as
\begin{equation}
\text{Riem}=R^a{}_{bcd}\pdv{}{x^a}\otimes\odif{x^b}\otimes\dd{x^c}\otimes\dd{x^d},
\end{equation}
with components
\begin{equation}
R^a{}_{bcd}=\pdif*{c}\Gamma^a_{bd}+\Gamma^e_{bd}\Gamma^a_{ec}-\pdif*{d}\Gamma^a_{bc}-\Gamma^e_{bc}\Gamma^a_{ed}.
\end{equation}
Then the Ricci tensor is defined as
\begin{equation}
\text{Ric}=R_{ab}\odif{x^a}\otimes\odif{x^b},
\end{equation}
which is just a contraction of the Riemann tensor in the first and third index, by which
\begin{equation}
R^a{}_{bad}=R_{bd}.
\end{equation}
Finally, we can define the Einstein tensor in terms of the Ricci tensor as
\begin{equation}
G_{ab}=R_{ab}-\frac{1}{2}Rg_{ab}.
\end{equation}

\subsection{Non-Zero Values of the Connection for the Geodesic Equations}
\label{apx:connectionvalues}
Recalling the definition for the Christoffel symbols from Eqn.~\eqref{eqn:christoffel},
\begin{equation}
\Gamma^\alpha{}_{\beta\delta}=\frac{1}{2}g^{\alpha \gamma}(g_{\gamma\beta;\delta}+g_{\gamma\delta;\beta}-g_{\beta\delta;\gamma}),
\end{equation}
we can define the non-zero values of the connection for the geodesic equation below.

$$\arraycolsep=25pt
\begin{array}{ccc}
\Gamma^r_{tt}=\dfrac{M(r-2M)}{r^3} & \Gamma^r_{rr}=-\dfrac{M}{r(r-2M)} & \Gamma^r_{\theta\theta}=-(r-2M)\\
\Gamma^r_{\phi\phi}=-(r-2M)\sin^2(\theta) & \Gamma^r_{rt}=\dfrac{M)}{r(r-2M)} & \Gamma^\theta_{r\theta}=\dfrac{1}{r}\\
\Gamma^\theta_{\phi\phi}=-\sin(\theta)\cos(\theta) & \Gamma^\phi_{r\phi}=\dfrac{1}{r} & \Gamma^\phi_{\theta\phi}=\dfrac{\cos(\theta)}{\sin(\theta)}
\end{array}$$

\subsection{Darwin Parameterisation: Expanded}
\label{apx:darwin}
Using the chain rule we may utilise the Darwin parameterisation in the following way:
\begin{equation}
\dv{t}{\chi}=\dv{t}{\tau}\dv{\tau}{r}\dv{r}{\chi}.
\end{equation}
Breaking this into parts, we get
\begin{align}
\dv{t}{\tau}&=\frac{r}{r-2M}\left(\frac{(p-2-2e)(p-2+2e)}{p(p-3-e^2)}\right)\\
&=\left(\frac{pM}{1+e\cos(\chi)}\left(\frac{1}{\frac{pM}{1+e\cos(\chi)}-2M} \right) \right)\left(\frac{(p-2-2e)(p-2+2e)}{p(p-3-e^2)}\right)\\
&=\left(\frac{pM}{1+e\cos(\chi)}\left(\frac{1}{\frac{pM-2M-2Me\cos(\chi)}{1+e\cos(\chi)}} \right) \right)\left(\frac{(p-2-2e)(p-2+2e)}{p(p-3-e^2)}\right)\\
&=\left(\frac{pM}{1+e\cos(\chi)}\left(\frac{1+e\cos(\chi)}{pM-2M-2Me\cos(\chi)} \right) \right)\left(\frac{(p-2-2e)(p-2+2e)}{p(p-3-e^2)}\right)\\
&=\left(\frac{p}{p-2-2e\cos(\chi)} \right)\left(\frac{(p-2-2e)(p-2+2e)}{p(p-3-e^2)}\right),\\
&\\
\dv{\tau}{r}&=\frac{1}{\sqrt{E^2-V}}\\
&=\left(\frac{(p-2-2e)(p-2+2e)}{p(p-3-e^2)}-\frac{r-2M}{r}\left(1+\frac{1}{r^2}\left(\frac{p^2M^2}{p-3-e^2} \right) \right)\right)^{-\frac{1}{2}}\\
\begin{split}
&=\left(\frac{(p-2-2e)(p-2+2e)}{p(p-3-e^2)}\right)^{-\frac{1}{2}}\\
&\qquad-\left(1-\frac{(1+e\cos(\chi))2M}{pM} \right)\left(1+\frac{p^2M^2(1-e\cos(\chi))^2}{p^2M^2(p-3-e^2)}\right)^{-\frac{1}{2}}
\end{split}\\
\begin{split}
&=\left(\frac{(p-2-2e)(p-2+2e)}{p(p-3-e^2)}\right)^{-\frac{1}{2}}\\
&\qquad-\left(\frac{((p-2+2e\cos(\chi))(p-2-e^2+2e\cos(\chi)+e^2\cos^2(\chi))}{p(p-3-e^2)} \right)^{-\frac{1}{2}}
\end{split}\\
&=\left(e^2\sin^2(\chi)\left(\frac{p-6-2e\cos(\chi)}{p(p-3-e^2)} \right)\right)^{-\frac{1}{2}},\\
&\\
\dv{r}{\chi}&=\left(\frac{pMe\sin(\chi)}{(1+e\cos(\chi))^2} \right).
\end{align}
Hence,
\begin{align}
\dv{t}{\chi}&=\dv{t}{\tau}\dv{\tau}{r}\dv{r}{\chi},\\
\begin{split}
&=\left(\frac{p}{p-2-2e\cos(\chi)} \right)\left(\frac{(p-2-2e)(p-2+2e)}{p(p-3-e^2)}\right)\\
&\qquad\times\left(e^2\sin^2(\chi)\left(\frac{p-6-2e\cos(\chi)}{p(p-3-e^2)} \right)\right)^{-\frac{1}{2}}\\
&\qquad\times\left(\frac{pMe\sin(\chi)}{(1+e\cos(\chi))^2} \right),
\end{split}\\
\begin{split}
&=p^2M(p-2+2e)^{\frac{1}{2}}(p-2-2e)^{\frac{1}{2}}(p-2-2e\cos(\chi))^{-1}\\
&\qquad\times(p-6-2e\cos(\chi))^{-\frac{1}{2}}(1+e\cos(\chi))^{-2}.
\end{split}
\end{align}
We also see that 
\begin{equation}
\dv{\phi}{\chi}=\dv{\phi}{\tau}\dv{\tau}{r}\dv{r}{\chi},
\end{equation}
where the latter two terms are the same as before, while the first term is known to be
\begin{equation}
\dv{\phi}{\tau}=\frac{L}{r^2}=\left(\frac{p^2M^2}{p-3-e^2} \right)^{\frac{1}{2}}\left(\frac{1+e\cos(\chi)}{pM} \right)^2.
\end{equation}
After substitution we get
\begin{align}
    \dv{\phi}{\chi}&=\left(\frac{L}{r^2}\right)\left(e^2\sin^2(\chi)\left(\frac{p-6-2e\cos(\chi)}{p(p-3-e^2)}\right)\right)^{-\frac{1}{2}}\left(\frac{pMe\sin(\chi)}{(1+e\cos(\chi))^2} \right),\\
    \begin{split}
    &=\left(\left(\frac{p^2M^2}{p-3-e^2} \right)\left(\frac{1+e\cos(\chi)}{pM} \right)^2\right)\left(\frac{(p(p-3-e^2))^{\frac{1}{2}}}{e\sin(\chi)(p-6-2e\cos(\chi))^{\frac{1}{2}}} \right)\\
    &\qquad\times\left(\frac{pMe\sin(\chi)}{(1+e\cos(\chi))^2} \right),
    \end{split}\\
    &=\left(p(p-6-2e\cos(\chi))^{-1}\right)^{\frac{1}{2}}.
\end{align}