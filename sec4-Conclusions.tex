Through developing an understanding of orbital motion, we have obtained an interesting result concerning approximations for parameter values that produce resonant orbits.
A plane of $p$ values has been extrapolated from a line of best fit for the $(p,x)$ values at $e=0$.
We can see from Table~\ref{table:fitvaluesAlle} that this predominantly linear relationship extends out for $0<e\leq 1$.

\begin{table}[!ht]
\begin{center}
\begin{tabular}{|c|c|c|c|}
\hline
     $e$ & $\alpha$ & $\beta$ & $\gamma$ \\
     \hline
     $0$ & $10.4048$ & $-5.13008$ & $-0.294414$ \\
     \hline
     $\quad 0.1\quad $ & $\quad 10.4084 \quad$ & $\quad -5.13152 \quad$ & $\quad -0.293166 \quad$ \\
     \hline
     $0.2$ & $10.419$ & $-5.13581$ & $-0.289436$\\
     \hline
     $0.3$ & $10.4367$ & $-5.14288$ & $-0.28327$\\
     \hline
     $0.4$ & $10.4613$ & $-5.15265$ & $-0.274738$\\
     \hline
     $0.5$ & $10.4926$ & $-5.16498$ & $-0.263937$\\
     \hline
     $0.6$ & $10.5306$ & $-5.17971$ & $-0.250984$\\
     \hline
     $0.7$ & $10.5748$ & $-5.19667$ & $-0.236012$\\
     \hline
     $0.8$ & $10.6252$ & $-5.21566$ & $-0.219165$\\
     \hline
     $0.9$ & $10.6814$ & $-5.23649$ & $-0.200594$\\
     \hline
     $1$ & $10.7432$ & $-5.25895$ & $-0.180455$\\
     \hline
\end{tabular}
\caption[Parameter values for the quadratic line of best fit for different $e$ values of the $(p,e,x)$ plane]{Parameter values for the line of best fit for different $e$ values of the plane in Fig.~\eqref{fig:pexplot}, where the line of best fit is of form $\alpha + \beta x +\gamma x^2$.}
\label{table:fitvaluesAlle}
\end{center}
\end{table}

This work could be taken further; given more time, an explicit form for the function $p(a)$ could be found so that our examination of the relationship between parameter values in resonance conditions can extend into a 3-dimensional problem.
Then it is possible to examine the relationship between $p$ and $a$ as well between $p$ and $e$ and $x$.
For example, the equation to be solved for a $(k,n)$ resonance where $e=0$ and $x=1$ is
\begin{equation}
    k\sqrt{\frac{4ap+p^{3/2}(p-4)-a^2\sqrt{p}}{2a+\sqrt{p}(p-3)}}=np^{1/4}\sqrt{\frac{p^2-4a\sqrt{p}+3a^2}{2a+\sqrt{p}(p-3)}}.
\end{equation}
Concerning the found method for extrapolation, it depends on a linear relationship out from $e=0$, but from Fig.~\eqref{fig:planeErr} it can be seen that a quadratic relationship still remains to be accounted for.
Error could be greatly reduced if a rudimentary understanding of this remaining quadratic element is achieved; for $a=0.9$ this would be a good start, but again, extending away from $a=0.9$ should be the end goal.

In the end, this thesis culminated in an analysis of resonant orbits, but there are a number of other features of orbital motion about black holes that would make for interesting directions in which a thesis could be taken.
To name but a few topics, the work done in this thesis does not account for any radiation reaction or self-force; these elements are already the focus of a number of research papers (see, e.g., \cite{SagoRadReac, GaltsovRadReac, MeentSelfForce}) and are the sole contributors to driving the evolution of an orbit.

A final point to mention is that, although the title of this thesis is ``Orbital Motion about a Black Hole", only bound orbits were examined. Entire papers are devoted to both plunging and scattered orbits (see, e.g., \cite{plungingExample, scatteredExample}), and if they were included in this one, it could have been quite possible that we may not have had the opportunity to discuss anything else.