\begin{titlepage}
    \begin{center}
        \vspace*{1cm}
        
        \Huge
        \textbf{Orbital Motion About a Black Hole}
        
        \vspace{1cm}
        
        Karl Coogan
        
        \vspace{1cm}

        \large
        The thesis is submitted to University College Dublin\\
in part fulfilment of the requirements for the degree of\\
\textbf{BSc Applied and Computational Mathematics}
        
        \vspace{1cm}
        
        \includegraphics[width=0.2\textwidth]{images/Universitycollegedublinlogo.png}
        
        \vspace{1cm}
        
        \large
        School of Mathematics \& Statistics\\
        University College Dublin\\
        \today

        \vspace{1cm}

        {\large Supervisor: Dr. Niels Warburton}
        
    \end{center}
\end{titlepage}
\thispagestyle{empty}
\begin{abstract}
\noindent
\centering
This thesis is concerned with orbital motion about black holes in both Schwarzschild and Kerr metrics, and is thematically an exploratory project at its core.
Current topics of interest involving black hole orbital motion include Extreme Mass Ratio Inspirals (where one black hole orbits another, much larger, black hole) and gravitational wave detection, where a greater understanding of the orbits can facilitate a deeper insight.
Analysing orbits in this thesis is done through numerical methods in Mathematica, where a large portion of the solving, plotting, and deriving takes place.
A number of derivations are done by hand when examining frequencies and reparameterising.
Once a method for calculating frequencies has been established, an analysis of resonant orbits takes place -- this topic is considered the main focus of this thesis outside of the fundamental black hole/general relativity knowledge that one must acquire before beginning to delve into other topics.
A discovery is made with regards to a linearity in the relationship between the orbital parameters $(p,e,x)$ (the semi-latus rectum or ``size" of the orbit, the eccentricity or non-circularity of the orbit, and the inclination angle from the equatorial plane at $\theta=\frac{\pi}{2}$ respectively), from which a method of estimating resonant $p$ values given values for $x$ and $e$ is created.
    \end{abstract}
    
    \vspace{1cm}
    
    \renewcommand{\abstractname}{Acknowledgements}
    \begin{abstract}
    \noindent
    I would like to thank Dr.~Niels Warburton for his support, which helped me navigate a research topic that initially I considered unknown territory; his guidance, which kept my goals clearly in reach and deterred any feelings of overwhelm; and his knowledge, without which it would not have been possible to produce a thesis of the quality presented here. I would also like to thank my fiancée, my family, and my peers for being my sounding boards and listening to me talk about black holes and orbital motion when I needed it, when I'm sure that they would have much rather spoken about anything else.
    \end{abstract}
    
    \clearpage

\thispagestyle{empty}
\null\vfill
\begin{dedication}
\settowidth\longest{\large\itshape in this universe. The more I examine the universe}
\centering
\parbox{\longest}{%
  \raggedright{\large\itshape%
   I believe in the ancient covenant. It is true that\\
   we emerged in the universe by chance, but\\
   the idea of chance is itself only a cover for\\
   our ignorance. I do not feel like an alien in this\\
   universe. The more I examine the universe\\
   and the details of its architecture, the more\\
   evidence I find that the universe in some\\
   sense must have known that we were coming...	\par\bigskip
  }   
  \raggedleft\Large\MakeUppercase{\textup{Freeman Dyson}}\par%
}
\end{dedication}

\vfill\vfill

\clearpage